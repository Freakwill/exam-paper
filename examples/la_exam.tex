%!TEX program = xelatex

\documentclass[12pt,a4paper]{ctexart}%
\usepackage[T1]{fontenc}%
\usepackage[utf8]{inputenc}%
\usepackage{lmodern}%
\usepackage{textcomp}%
\usepackage{lastpage}%
\usepackage{mathrsfs, amsfonts, amsmath, amssymb}%
\usepackage{enumerate}%
\usepackage{analysis, algebra}%
\usepackage{exampaper}%
\usepackage{fancyhdr}%
\usepackage{geometry}%
\usepackage{ragged2e}%
\usepackage{multirow}%
%
\geometry{left=3.3cm,right=3.3cm,top=2.3cm,foot=1.5cm}%
\pagestyle{fancy}%
\cfoot{\footnotesize{第~\thepage~页~(共~\pageref{LastPage}~页)}}%

\renewcommand{\headrulewidth}{0pt}%
%
\begin{document}%
\begin{center}%
\Large{\textbf{浙江工业大学之江学院第 2017/2018 学年\\第 2 学期试卷 I}}%
\end{center}%
\begin{center}%
课程\myline[5.8cm]~班级\myline[5.8cm]\\%
姓名\myline[3.5cm]~学号\myline[3.5cm]~教师姓名\myline[3.5cm]%
\end{center}%
\begin{center}%
\begin{tabular}{|c|c|c|c|c|c|}%
\hline%
\sws{题号}&\sws{一}&\sws{二}&\sws{三}&\sws{四}&\sws{总评}\\%
\hline%
\multirow{2}{*}{计分}&&&&&\\%
&&&&&\\%
\hline%
\end{tabular}%
\end{center}%
\noindent 一、填空题 (每空 2 分, 共 20 分):%
\begin{enumerate}[1)]%

\item%
$\begin{bmatrix}
0 & x & 1 & 0\\
y & 3 & 0 & 1\\
0 & 2 & z & 0
\end{bmatrix}
$经初等变换$r_1\leftrightarrow r_2, r_1 + 2r_3$变成\autolenunderline{};%
\item%
设$A=\begin{bmatrix}
x & 1 & x\\
0 & y & 0
\end{bmatrix}
$, 当\autolenunderline{}时$A$的秩为1, 否则$A$的秩为\autolenunderline{}, $A$的秩一定不为\autolenunderline{};%
\item%
设$A=\begin{bmatrix}
x& 0 & 0 \\
-y& x & 1 \\
0 & 1 & z
\end{bmatrix}
$, 则$A^T=$\autolenunderline{}, $A$是对称矩阵当且仅当\autolenunderline{};%
\item%
设行列式$\begin{vmatrix}
x & 1 & 1 \\
1 & x & -1 \\
2 & -1 & 1
\end{vmatrix}
=0$, 则$x=$\autolenunderline{} 或\autolenunderline{};%

\item%
设矩阵$A=\begin{bmatrix}
1 & 1 & 0\\
0 & 1 & 1
\end{bmatrix}
$, 则$A^TA=$\autolenunderline{}, $AA^T=$\autolenunderline{};%
\end{enumerate}%


%
\noindent 二、判断题 (每空 2 分, 共 10 分):%
\begin{enumerate}[1)]%
\item%
第三类初等行(列)变换不会改变矩阵的行列式;~~\mypar{}%
\item%
$n$行列式最多有$n^2$个求和项;~~\mypar{}%
\item%
分块矩阵乘积的最终结果和如何分块(包括不分块, 只要分块方案可行)是无关的;~~\mypar{}%
\item%
线性代数在数学内部、自然科学、社会科学、生产生活方方面面都有广泛应用;~~\mypar{}%
\item%
矩阵乘积的行列式是行列式的乘积, 即$|AB|=|A||B|$;~~\mypar{}%
\end{enumerate}%


%
\noindent 三、选择题 (每空 2 分, 共 10 分):%
\begin{enumerate}[1)]%
\item%
下面哪些问题可以用初等变换做;~~\mypar{}\\
(A) 求出标准型~~(B) 求行列式~~(C) 解线性方程组~~(D) 以上都是%
\item%
初等变换可能改变矩阵的什么;~~\mypar{}\\
(A) 行列式~~(B) 标准型~~(C) 秩~~(D) 可逆性%
\item%
下面哪个关于矩阵的比喻比较贴切;~~\mypar{}\\
(A) 一幅素描~~(B) 一张Excel表单~~(C) 俄罗斯方块~~(D) 9宫格%
\item%
3阶行列式展开一共几个求和项;~~\mypar{}\\
(A) 3~~(B) 6~~(C) 9~~(D) 12%
\item%
下面哪种行列式可能不为0;~~\mypar{}\\
(A) 0行列式~~(B) 有0行(列)~~(C) 对角行列式~~(D) 某一行(列)是另一行(列)的常数倍%
\end{enumerate}%

%
\noindent 四、计算题 (每题 10 分, 共 60 分):%
\begin{enumerate}[1)]%
\item%
计算行列式$\begin{vmatrix}%
-6&-3&5\\%
6&-5&5\\%
2&-4&5%
\end{vmatrix}$.

\clearpage%
\item%
计算$-1\begin{bmatrix}
-3 & -1 & 1\\
2 & 3 & -4\\
2 & -1 & 4
\end{bmatrix} -1\begin{bmatrix}
-2 & -4 & -2\\
-3 & 0 & -2\\
4 & -3 & 4
\end{bmatrix}$.

\vspace{10cm}%
\item%
计算$\begin{bmatrix}
3 & 1 & 2 & 1\\
-2 & 4 & 4 & 3
\end{bmatrix}\begin{bmatrix}
1 & -3 & 1\\
1 & -1 & 4\\
1 & 1 & 1\\
4 & 3 & -3
\end{bmatrix}$.

\clearpage%
\item%
用分块矩阵的方法计算$\begin{bmatrix}
-3 & 4 & 3 & 0\\
0 & 0 & 0 & 0\\
0 & 0 & 0 & 1
\end{bmatrix}\begin{bmatrix}
0 & -5\\
0 & -2\\
-4 & -1\\
0 & 0
\end{bmatrix}$. (用线条注明分块方案)

\vspace{10cm}%
\item%
用初等变换将矩阵$\begin{bmatrix}
-4 & 0 & -4 & 20\\
-1 & -3 & 14 & -4\\
-2 & 2 & -10 & 10
\end{bmatrix}$变成标准型.

\clearpage%
\item%
求解线性方程组$\left\{\begin{array}{rl}
- x_{1} + 5x_{2} + 4x_{3}& =2 \\
- 2x_{1} + 7x_{2} + 17x_{3}& =3 \\
- 3x_{1} + 15x_{2} + 14x_{3}& =-1 \\
\end{array}\right.$.

\vspace{10cm}%
\end{enumerate}%
\end{document}