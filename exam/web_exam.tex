%!TEX program = xelatex

\documentclass[12pt,a4paper]{ctexart}%
\usepackage[T1]{fontenc}%
\usepackage[utf8]{inputenc}%
\usepackage{lmodern}%
\usepackage{textcomp}%
\usepackage{lastpage}%
\usepackage{mathrsfs, amsfonts, amsmath, amssymb}%
\usepackage{enumerate}%
\usepackage{analysis, algebra}%
\usepackage{exampaper}%
\usepackage{fancyhdr}%
\usepackage{geometry}%
\usepackage{ragged2e}%
\usepackage{multirow}%
%
\geometry{left=3.3cm,right=3.3cm,top=2.3cm,foot=1.5cm}%
\pagestyle{fancy}%
\chead{\textbf{浙 江 工 业 大 学 之 江 学 院 考 试 命 题 纸}}%
\cfoot{\footnotesize{第~\thepage~页~(共~\pageref{LastPage}~页)}}%
\renewcommand{\headrulewidth}{0pt}%
%
\begin{document}%
\begin{center}%
\Large{\textbf{浙江工业大学之江学院\\期终试卷样卷}}%
\end{center}%

\begin{center}%
\begin{tabular}{lclc}%
二级学院:&理学院&专业名称:&信计专业\\%
课程名称:&爬虫与Web数据挖掘&课程代码:&30800500\\%
主讲教师:&宋丛威&&\\%
\end{tabular}%
\end{center}%

\begin{center}%
\begin{tabular}{|c|c|c|c|c|c|}%
\hline%
\sws{题序}&\sws{一}&\sws{二}&\sws{三}&\sws{四}&\sws{总分}\\%
\hline%
\multirow{2}{*}{计分}&&&&&\\%
&&&&&\\%
\hline%
\end{tabular}%
\end{center}%
\thispagestyle{plain}%
\noindent 一、填空题 (每空 2 分, 共 20 分):%
\begin{enumerate}[1)]%
\item%
向网站发出请求的方法有\autolenunderline{}、\autolenunderline{};%
\item%
写出匹配国内13位电话号码的正则表达式\autolenunderline{};%
\item%
写出至少两种常见HTML标签名\autolenunderline{}、\autolenunderline{};%
\item%
用于存储网页数据的数据库有\autolenunderline{}、\autolenunderline{};%
\item%
BeautifulSoup 常用HTML标签搜索方法有\autolenunderline{}、\autolenunderline{};%
\item%
统一资源定位符简称\autolenunderline{};%
\end{enumerate}%


%
\noindent 二、判断题 (每空 2 分, 共 10 分):%
\begin{enumerate}[1)]%
\item%
Python 有大量优秀的第三方库进行网络编程;~~\mypar{}%
\item%
Python 只适合数据分析,不适合设计网络爬虫;~~\mypar{}%
\item%
requests与beautifulSoup的组合经常与用于 Python 的网络爬虫设计;~~\mypar{}%
\item%
下载电影用迅雷比用 Python 写爬虫更方便;~~\mypar{}%
\item%
beautifulSoup 其实是一个 HTML 语言的解析器,将 HTML 翻译成 Python 的数据结构;~~\mypar{}%
\end{enumerate}%


%
\noindent 三、选择题 (每空 2 分, 共 10 分):%
\begin{enumerate}[1)]%
\item%
号称 Python 网络爬虫编程绝配是一组合?~~\mypar{}\\
(A) 美女与野兽~~(B) 牛奶与巧克力~~(C) requests 与 bs4~~(D) urllib 与 re%
\item%
下面哪一种是常用的反爬策略?~~\mypar{}\\
(A) 更换电池~~(B) 更换WiFi~~(C) 增加头信息~~(D) 增加元信息%
\item%
下面哪个是Python的网络爬虫框架;~~\mypar{}\\
(A) beautifulSoup~~(B) requests~~(C) re~~(D) scrapy%
\item%
下面哪个网络状态码表示请求成功;~~\mypar{}\\
(A) 400~~(B) 404~~(C) 501~~(D) 200%
\item%
下面哪一件事情是Python网络编程无法直接实现的?~~\mypar{}\\
(A) 下载一部小说~~(B) 瘫痪金融网络~~(C) 入侵FBI~~(D) 获得友情或爱情%
\end{enumerate}%


%
\noindent 四、计算题 (共 60 分):%
\begin{enumerate}[1)]%
\item% 
(10 分)
下载指定网站中的一部网络小说.
\item%
(10 分)
下载新闻,并统计其中的文字或词语的频数.
\item%
(40 分)
利用爬虫框架,开发一款爬虫软件.
\end{enumerate}%

%
\end{document}