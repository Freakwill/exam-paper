%!TEX program = xelatex

\documentclass[12pt,a4paper]{ctexart}%
\usepackage[T1]{fontenc}%
\usepackage[utf8]{inputenc}%
\usepackage{lmodern}%
\usepackage{textcomp}%
\usepackage{lastpage}%
\usepackage{mathrsfs, amsfonts, amsmath, amssymb}%
\usepackage{enumerate}%
\usepackage{analysis, algebra}%
\usepackage{exampaper}%
\usepackage{fancyhdr}%
\usepackage{geometry}%
\usepackage{ragged2e}%
\usepackage{multirow}%
%
\geometry{left=3.3cm,right=3.3cm,top=2.3cm,foot=1.5cm}%
\pagestyle{fancy}%
\chead{\textbf{浙 江 工 业 大 学 之 江 学 院 考 试 命 题 纸}}%
\cfoot{\footnotesize{第~\thepage~页~(共~\pageref{LastPage}~页)}}%
\renewcommand{\headrulewidth}{0pt}%
%
\begin{document}%
\normalsize%
\begin{center}%
\Large{\textbf{浙江工业大学之江学院第 2018/2019 学年\\第 1 学期试卷}}%
\end{center}%
\noindent%
二级学院:\hspace{6cm}专业名称:\hspace{6cm}\\课程名称:\hspace{6cm}课程代码:\hspace{6cm}\\主讲教师:\hspace{6cm}%
\begin{center}%
\begin{tabular}{|c|c|c|c|c|c|}%
\hline%
\sws{题号}&\sws{一}&\sws{二}&\sws{三}&\sws{四}&\sws{总评}\\%
\hline%
\multirow{2}{*}{计分}&&&&&\\%
&&&&&\\%
\hline%
\end{tabular}%
\end{center}%
\noindent 一、填空题 (每空 2 分, 共 20 分):%
\begin{enumerate}[1)]%
\item%
Python 内置丑了吧唧的 IDE 是\autolenunderline{idle}; 公认最好的 IDE 是 \autolenunderline{PyCharm}.%
\item%
Python 网络编程堪称绝配的两个第三方库: 、\autolenunderline{bs4};%
\item%
说程序设计体现 Python 简洁明了的风格的形容词是\autolenunderline{pythonic};%
\item%
文件操作结束时候要执行\autolenunderline{close}命令;%
\item%
C 语言中的结构体类似于 Python 中的$\autolenunderline{dict}$类型;%
\item%
Python 的吉祥物是\autolenunderline{python};%
\item%
循环语句中,Python 的控制命令有: \autolenunderline{break}、\autolenunderline{continue};%
\item%
Python脚本文件后缀是\autolenunderline{.py};%
\end{enumerate}%


%
\noindent 二、判断题 (每空 2 分, 共 10 分):%
\begin{enumerate}[1)]%
\item%
Python 比 C、Java 容易学;~~(\true)%
\item%
Python 运行速度比较慢,但开发效率高;~~(\true)%
\item%
Python 真正实现了"万物皆对象"的程序设计理念;
~~(\true)%
\item%
学习 Python 的学生比不学习的更有可能获得幸福和快感,也更长寿;~~(\true)%
\item%
Python 有 case 语句;~~(\false)%
\end{enumerate}%


%
\noindent 三、选择题 (每空 2 分, 共 10 分):%
\begin{enumerate}[1)]%
\item%
存储个人信息时,最适合采用什么数据类型?~~(\mypar{B})\\
(A) 元组~~(B) 字典~~(C) 列表~~(D) 集合%
\item%
文件操作完毕后,应该;~~(\mypar{D})\\
(A) 直接退出程序~~(B) 关机~~(C) 拔电源~~(D) 关闭文件对象%
\item%
下面哪个是Python创建函数的关键词;~~(\mypar{A})\\
(A) def~~(B) func~~(C) class~~(D) function%
\item%
下面哪个起到流程控制作用的关键词不是Python的关键词;~~(\mypar{D})\\
(A) break~~(B) continue~~(C) return~~(D) goto%
\item%
Python 的吉祥物是;~~(\mypar{D})\\
(A) 一只袋鼠~~(B) 一只企鹅~~(C) 一只章鱼~~(D) 一条蟒蛇%
\end{enumerate}%


%
\noindent 四、计算题 (每题 10 分, 共 60 分):%
%
\end{document}